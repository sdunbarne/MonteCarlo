%%% -*-LaTeX-*-
%%% montecarlo.tex.orig
%%% Prettyprinted by texpretty lex version 0.02 [21-May-2001]
%%% on Mon Aug 22 08:54:48 2022
%%% for Steve Dunbar (sdunbar@family-desktop)

\documentclass[12pt]{article}

\input{../../../../etc/macros} %% \input{../../../../etc/mzlatex_macros}
\input{../../../../etc/pdf_macros}

\bibliographystyle{plain}

\begin{document}

\myheader \mytitle

\hr

\sectiontitle{Markov Chain Monte Carlo}

\hr

\usefirefox

\hr

% \visual{Study Tip}{../../../../CommonInformation/Lessons/studytip.png}
% \section*{Study Tip}

% \hr

\visual{Rating}{../../../../CommonInformation/Lessons/rating.png}
\section*{Rating} %one of
% Everyone: contains no mathematics.
% Student: contains scenes of mild algebra or calculus that may require guidance.
% Mathematically Mature: may contain mathematics beyond calculus with proofs.
Mathematicians Only:  prolonged scenes of intense rigor.

\hr

\visual{Section Starter Question}{../../../../CommonInformation/Lessons/question_mark.png}
\section*{Section Starter Question}

In statistical mechanics from physics, what is the Boltzmann
distribution and what does it represent?

\hr

\visual{Key Concepts}{../../../../CommonInformation/Lessons/keyconcepts.png}
\section*{Key Concepts}

\begin{enumerate}
    \item
        With Markov chain Monte Carlo, to get a sample, select \( s_1
        \in \mathcal{X} \) arbitrarily.  If \( s_{k-1} = i \), select \(
        s_k = j \) with probability \( p_{ij} \).  By the Fundamental
        Theorem for Markov Chains and the Strong Law of Large Numbers
        for any \( k > 1 \), the resulting sequence \( s_1, s_2, \dots \)
        has, as \( M \to \infty \),
        \[
            \frac{\card{\setof{k}{k \le M, s_k = j}}}{M} \to \pi_j
        \] with probability \( 1 \).
    \item
        The heart of the MCMC method is the inverse problem:  Given a
        distribution \( \pi \) on a finite state space, find an
        irreducible, aperiodic Markov chain that is stationary on \( \pi
        \).  The solution to this inverse problem is the \defn{Metropolis
        algorithm}.
    \item
        The formal statement of the Metropolis algorithm for a suitable
        proposed transition is:  For an arbitrary \( x \in \mathcal{X} \)
        define the transition to a configuration \( x^{\star} \).
        \begin{enumerate}
            \item
                Select \( x' \) according to the proposed transition.
            \item
                If \( E(x') \le E(x) \) or equivalently \(
                \operatorname{Boltz}
                (x') \ge
                \operatorname{Boltz}
                (x) \), let \( x^{\star} = x' \).  In other words,
                always move to a lower energy (higher probability)
                configuration.
            \item
                If \( E(x') > E(x) \) or equivalently \(
                \operatorname{Boltz}
                (x') <
                \operatorname{Boltz}
                (x') \), let \( x^{\star} = x' \) with probability
                \[
                    \frac{%
                    \operatorname{Boltz}
                    (x')}{%
                    \operatorname{Boltz}
                    (x)} = \EulerE^{-\Delta E/kT}.
                \]
            \item
                Otherwise, \( x^{\star} = x' \).
        \end{enumerate}
    \item
        %% Need notation fix in this item
        For the Metropolis-Hastings algorithm, given a distribution \(
        \pi \) to sample, select any transition probability matrix \( P
        = (p_{ij}) \) on the state space.  Unlike the Metropolis
        algorithm, \( P \) does not need to be symmetric.  Define the
        transition probability matrix \( Q= (q_{ij}) \) by
        \[
            Q_{ij} =
            \begin{cases}
                p_{ij} \alpha_{ij} \text{ if \( i \ne j \) } \\
                1 - \sum\limits_ {k \ne i} q_{ik} \text{ if \( i = j \)
                }
            \end{cases}
        \] where \( \alpha_{ij} \) is given by
        \[
            \alpha_{ij} = \frac{s_{ij}}{1 + \frac{\pi_i}{\pi_j} \frac{p_
            {ij}}{p_{ji}}}.
        \] The values \( s_{ij} \) can be quite general so long as \( s_
        {ij} = s_{ji} \) for all \( i \) and \( j \) and \( 0 \le \alpha_
        {ij} \le 1 \).
\end{enumerate}

\hr

\visual{Vocabulary}{../../../../CommonInformation/Lessons/vocabulary.png}
\section*{Vocabulary}
\begin{enumerate}
    \item
        The \defn{Markov chain Monte Carlo} (MCMC) method depends on the
        observation that if \( \pi \) is the stationary distribution for
        an irreducible, aperiodic Markov chain, then we can use the
        Markov chain to sample from \( \pi \).
    \item
        The heart of the MCMC method is the inverse problem:  Given a
        distribution \( \pi \) on a finite state space, find an
        irreducible, aperiodic Markov chain that is stationary on \( \pi
        \).  Two solutions to this inverse problem are the \defn{Metropolis
        algorithm} and the \defn{Metropolis-Hastings algorithm}.
    \item
        In a 1970 paper, the statistician W. K. Hastings cast the
        Metropolis algorithm as a general purpose sampling algorithm,
        distilling the Metropolis algorithm to mathematical essentials.
        Hastings's generalization of the Metropolis algorithm is the
        \defn{Metropolis-Hastings algorithm}.
    \item
        Suppose the Markov chain is \defn{reversible}, so \( \pi(x) P(x,y)
        = \pi(y) P(y,x) \).  In words, reversibility means that at
        equilibrium the probability of going from state \( x \) to state
        \( y \) is the same as going from \( y \) to \( x \).
\end{enumerate}

\hr

\section*{Notation}
\begin{enumerate}
    \item
        \( \pi \) -- the stationary distribution for an irreducible,
        aperiodic Markov chain
    \item
        \( \mathcal{X} \) -- state space for Markov chain
    \item
        \( s_1 \) -- arbitrary element of the state space
    \item
        \( n > 1 \) -- arbitrary step in Markov chain sample
    \item
        \( s_{n-1}, s_n \) -- successive steps in Markov chain sample
    \item
        \( p_{ij} \) -- transition probability from state \( i \) to
        state \( j \)
    \item
        \( M, m \) -- large sample sizes in Markov chain sample
    \item
        \( f \) -- arbitrary function on the state space
    \item
        \( E(\cdot) \) -- energy function
    \item
        \( T \) -- temperature
    \item
        \( \tilde{\mathcal{X}} \) be any sample of configurations
        selected \emph{with replacement} from \( \mathcal{X} \)
    \item
        \( \card{\tilde{\mathcal{X}}} = \tilde{N} \)
    \item
        \( N_{x} \) -- the number of occurrences of \( x \) in \( \tilde
        {\mathcal{X}} \)
    \item
        \( z = \sum\limits_{x' \in \mathcal{X}} \EulerE^{-E(x')/kT} \)
        -- the partition function.
    \item
        \( k \) -- Boltzman constant
    \item
        \(
        \operatorname{Boltz}
        (x') \) -- Boltzman probability density
    \item
        \( P(x,x') \) -- transition probability matrix on \( \mathcal{X}
        \)
    \item
        \( P = (p_{ij}) \) -- transition probability matrix on the state
        space
    \item
        \( Q \) -- alternative transition probability matrix from
        Metropolis or Metropolis Hastings algorithm
    \item
        \( L^2(\pi) \) -- the set of mappings \( \{\vec{g} :  \mathcal{X}
        \to \Reals \} \) with inner product \( \langle \vec{g},\vec{h}
        \rangle = \sum_{x}g(x) h(x) \pi(x) \)
    \item
        \( \beta_i \) -- real eigenvalues
    \item
        \( \vec{\psi}_i \) -- eigenvectors
    \item
        \( (x_{i1}, x_{i2}) \) for \( i = 1,2,3 \) -- positions for
        three particles each randomly located in the square \( [0,1]^2
        \subset \Reals^2 \)
    \item
        \( C \) and \( D \) -- fixed positive constants

\end{enumerate}

\visual{Mathematical Ideas}{../../../../CommonInformation/Lessons/mathematicalideas.png}
\section*{Mathematical Ideas}

%% \subsection*{Basic Ideas of MCMC}

The \defn{Markov chain Monte Carlo} (MCMC)%
\index{Markov chain Monte
  Carlo}
method depends on the observation:
\begin{quotation}
    If \( \pi \) is the stationary distribution for an irreducible,
    aperiodic Markov chain, then we can use the Markov chain to sample
    from \( \pi \).
\end{quotation}
To get a sample, select \( s_1 \in \mathcal{X} \) arbitrarily.  For any \(
n > 1 \), if \( s_{n-1} = i \), select \( s_n = j \) with probability \(
p_{ij} \).  By the Fundamental Theorem for Markov Chains and the Strong
Law of Large Numbers the resulting sequence \( s_1, s_2, \dots \) has,
as \( M \to \infty \),
\[
    \frac{\card{\setof{n}{n \le M, s_n = j}}}{M} \to \pi_j
\] with probability \( 1 \).

Any large but finite part of the sequence approximates a sample from \(
\pi \).  Often, one discards the first \( m \) terms of the sequence and
uses the ``tail'' of the sequence, \( s_{m+1}, s_{m+2}, \dots, s_M \) as
the sample.

Samples from \( \pi \), however obtained, provide a way to approximate
properties of \( \pi \).  Suppose \( f \) is a real-valued function on
the state space \( \mathcal{X} \).  To approximate the expected value
\[
    \E{f} = \sum\limits_{\nu=1}^N f(x_\nu) \pi_\nu
\] select a sample \( s_1, s_2, \dots s_M \) from \( \pi \) and the
ergodic theorem guarantees
\[
    \frac{1}{M} \sum\limits_{\nu=1}^M f(s_\nu) \to \E{f}
\] as \( M \to \infty \) with convergence rate \( O(M^{-1/2}) \).

Given a reasonably sized transition probability matrix for an
irreducible, aperiodic Markov chain, it is a standard to find its
stationary distribution.  The heart of the MCMC method is the inverse
problem:
\begin{quotation}
    Given a distribution \( \pi \) on a finite state space, find an
    irreducible, aperiodic Markov chain stationary on \( \pi \).
\end{quotation}
Two solutions to this inverse problem are the Metropolis algorithm and
the Metropolis-Hastings algorithm.%
\index{Metropolis algorithm}
\index{Metropolis-Hastings algorithm}
The Metropolis algorithm was motivated by the desire to understand
properties of the \emph{Boltzmann distribution} from statistical
mechanics, the branch of physics concerned with the average behavior of
large systems of interacting particles.

\subsection*{A Motivational Example from Statistical Mechanics}

Consider a very large but conceptually finite state space \( \mathcal{X}
\), also called a configuration space \( \mathcal{X} \),%
\index{configuration space}
an energy function \( E(\cdot) \) on \( \mathcal{X} \) and a fixed
temperature \( T \).  Let \( \tilde{\mathcal{X}} \) be any sample of
configurations selected \emph{with replacement} from \( \mathcal{X} \).
It is desirable to let \( \tilde{\mathcal {X}} \) be larger than \(
\mathcal{X} \).  The goal is to change \( \tilde{\mathcal{X}} \) so that
it becomes approximately a sample from the Boltzmann distribution.%
\index{Boltzmann distribution}
Suppose \( \card{\tilde{\mathcal{X}}} = \tilde{N} \) and let \( N_{x} \)
denote the number of occurrences of \( x \) in \( \tilde{\mathcal{X}} \).
To say the sample approximates the Boltzmann distribution means
\[
    \frac{N_{x}}{N} = \frac{\EulerE^{-E(x)/kT}}{z}
\] for the proportionality or normalizing constant \( z \).  The
denominator
\[
    z = \sum\limits_{x' \in \mathcal{X}} \EulerE^{-E(x')/kT}
\] is called the partition function.%
\index{partition function}
In any realistic setting, the partition function is intractable, usually
due to the size of \( \mathcal{X} \).  However, for any two
configurations \( x \) and \( x' \)
\begin{equation}
    \label{eq:montecarlo:boltzmandiff} \frac{N_{x'}}{N_{x}} = \frac{\EulerE^
    {-E(x')/kT}}{\EulerE^{-E(x)/kT}} = \EulerE^{-(E(x')-E(x))/kT} =
    \EulerE^{-\Delta E/kT}.
\end{equation}
This ratio does not depend on the partition function.

Now start from any irreducible, aperiodic Markov chain on \( \mathcal{X}
\).  Let the transition probability from configuration \( x \) to \( x' \)
be \( P_{x, x'} \).  Also, let the transition probability matrix be
symmetric, \( P_{x, x'} = P_{x', x} \) so the proposed Markov chain is
reversible.  Allow transitions from configurations with high energy \( E
(x') \) to lower energy \( E(x) \) when they are proposed.  Any downhill
transition is allowed, the choice is based on convenience and
naturalness.  The number of times this occurs is \( P_{x', x} N_{x'} \).

To reach equilibrium with a complete distribution of energies, not just
the low energies achieved by flowing downhill, occasionally allow uphill
transitions from configurations with low energy \( E(x) \) to high
energy \( E(x') \) with some probability \( \Prob{x \to x'} \).  The
number of times the uphill transition occurs is \( P_{x, x'} N_{x}\Prob{x
\to x'} \).  Since the Markov chain is reversible, the transition
probability is symmetric and \( P_{x', x} = P_{x, x'} \).  The net flux
from configurations with high energy \( E(x') \) to low energy \( E(x) \)
is
\begin{equation}
    \label{eq:montecarlo:netflux} P_{x', x} N_{x'} - P_{x,x'}N_x \Prob{x
    \to x'} = P_{x, x'} \left( N_{x'} - N_{x} \Prob{x \to x'}\right).
\end{equation}

In equilibrium, so that the distribution of energies in \( \tilde{\mathcal
{X}} \) reflects the Boltzmann distribution very well, the net energy
flux should be zero.  Using equation~\eqref{eq:montecarlo:boltzmandiff}
implies that the uphill probability must be
\[
    \Prob{x \to x'} = \frac{N_{x'}}{N_{x}} = \EulerE^{-\Delta E/kT}.
\]

This modified Markov chain process will drive an arbitrary distribution
of energies toward the Boltzmann distribution for the following reasons.
Suppose the distribution has too many configurations with high energy \(
E(x') \) compared with configurations with the low energy \( E(x) \),
that is,
\[
    \frac{N_{x'}}{N_{x}} > \EulerE^{-\Delta E/kT}.
\] In this case, the net flux in equation~%
\ref{eq:montecarlo:netflux} is positive and there will be more
transitions from configurations with high energy to those with low
energy than in the other direction.  The distribution of energies in \(
\tilde{\mathcal {X}} \) will move properly towards the Boltzmann
distribution.  Repeating the process will produce a set of
configurations whose distribution of energies approximates the Boltzmann
distribution.  Understanding the rate of convergence is left for further
investigation.

The following is a formal statement of the Metropolis algorithm for a
suitable proposed transition.  For an arbitrary \( x \in \mathcal{X} \)
define the transition to a configuration \( x^{\star} \).
\begin{enumerate}
    \item
        Select \( y \) according to the proposed transition.
    \item
        If \( E(y) \le E(x) \) or equivalently \(
        \operatorname{Boltz}
        (y) \ge
        \operatorname{Boltz}
        (y) \), let \( x^{\star} = y \).  In other words, always move to
        a lower or equal energy (equivalently, a greater or equal
        probability) configuration.
    \item
        If \( E(y) > E(x) \) or equivalently \(
        \operatorname{Boltz}
        (y) <
        \operatorname{Boltz}
        (x') \), let \( x^{\star} = y \) with probability
        \[
            \frac{%
            \operatorname{Boltz}
            (y)}{%
            \operatorname{Boltz}
            (x)} = \EulerE^{-\Delta E/kT}.
        \] In other words, move to a higher energy (lesser probability)
        configuration with probability \( \EulerE^{-\Delta E/kT} \).
    \item
        Otherwise, \( x^{\star} = y \).
\end{enumerate}
This process defines an irreducible, aperiodic Markov chain on the
configuration space \( \mathcal{X} \).  The ratio is crucial to the
computational utility of the Metropolis algorithm because it avoids the
intractable partition function.  The steps of the chain are easily
computable, or at least as easy as the proposal transition.  In many
settings, \( \Delta E = E(x') - E(x) \) is simple to compute, often it
is independent of \( \card{\mathcal{X}} \).  The Markov chain defined by
the Metropolis algorithm can be implemented without knowing the entire
transition matrix.

\subsection*{Abstraction of the Metropolis Algorithm}

The Metropolis algorithm%
\index{Metropolis algorithm}
creates an easily computed Markov chain that is stationary on the
desired distribution \( \pi \).  The algorithm requires only the
relative weights of the limiting distribution, not the probabilities,
thereby avoiding the possibly intractable normalizing constant \( z \).

Let \( \mathcal{X} \) be a potentially large but finite state space and \(
\pi(x) \) a probability mass vector on \( \mathcal{X} \), perhaps
specified only up to an unknown normalizing constant.  Let \( P \) be a
transition probability matrix on \( \mathcal{X} \) with \( P(x,x') > 0 \)
if and only if \( P(x',x) > 0 \).  This is a weaker requirement than
reversibility.  Using parentheses instead of integer subscripts to index
entries is a slight change of notation for matrices.  Increasing the
analogy to the previous example from statistical mechanics, along with a
connection to spectral theory used later motivates the change in
notation.  At the start \( P \) is unrelated to \( \pi \).  The \defn{Metropolis
algorithm} changes \( P \) to a new transition probability matrix \( Q(x,x')
\) with stationary distribution \( \pi \).  Define the \emph{acceptance
ratio} as
\[
    A(x,x') = \frac{\pi(x') P(x',x)}{\pi(x) P(x,x')}.
\] The potential lack of a normalizing constant for the distribution \(
\pi \) does not matter since it cancels in the acceptance ratio.  The
Metropolis algorithm changes \( P \) by
\[
    Q(x,x') =
    \begin{cases}
        P(x,x') & x \ne x', A(x,x') \ge 1 \\
        P(x,x')A(x,x') & x \ne x', A(x,x') < 1\ \\
        P(x,x') + \sum\limits_{z:A(x,z) < 1} P(x,z)(1 - A(x,z)) & x =
        x'.
    \end{cases}
\]

The Metropolis algorithm has a simple interpretation:  from \( x \),
choose \( x' \) with probability \( P(x,x') \) as in the original Markov
chain.  Then if the acceptance ratio \( A(x,x') \ge 1 \), move to \( x' \);
if the acceptance ratio \( A(x,x') < 1 \), flip a coin with success
probability \( A(x,x') \) and move to \( x' \) if success occurs; in
other cases stay at \( x \).

The new chain satisfies
\[
    \pi(x) Q(x,x') = \pi(x') Q(x',x)
\] and thus
\[
    \sum\limits_{x \in \mathcal{X}} \pi(x) Q(x,x') = \sum\limits_{x \in
    \mathcal{X}} \pi(x') Q(x',x) = \pi(x') \sum\limits_{x \in \mathcal{X}}
    Q(x',x) = \pi(x').
\] Therefore, \( \pi \) is a left eigenvector with eigenvalue \( 1 \).
If the Markov chain is connected, then the Fundamental Theorem of Markov
Chains is in effect.  After many steps of the chain, it should have
converged to a stationary distribution.  That is, the chance of being at
\( x \) is approximately \( \pi(x) \), no matter what the starting
state.

\subsection*{Metropolis-Hastings Algorithm}

In a 1970 paper, the statistician W. K. Hastings cast the Metropolis
algorithm as a general purpose sampling algorithm, distilling the
Metropolis algorithm to mathematical essentials.  The following is
Hastings's generalization of the Metropolis algorithm, known as the
\defn{Metropolis-Hastings algorithm}.%
\index{Metropolis-Hastings algorithm}
Given a distribution \( \pi \) to sample, select any transition
probability matrix \( P = (p_{ij}) \) on the state space.  Here the
notation for matrix entries reverts again to subscripts since the
application is typically for matrices on finite dimensional spaces.
Unlike the Metropolis algorithm, \( P \) does not need to be symmetric.
Define the transition probability matrix \( Q= (q_{ij}) \) by
\[
    Q_{ij} =
    \begin{cases}
        p_{ij} \alpha_{ij} \text{ if \( i \ne j \) } \\
        1 - \sum\limits_{\nu \ne i} p_{i\nu} \alpha_{i \nu} \text{ if \(
        i = j \) }
    \end{cases}
\] where \( \alpha_{ij} \) is given by
\[
    \alpha_{ij} = \frac{s_{ij}}{1 + \frac{\pi_i}{\pi_j} \frac{p_{ij}}{p_
    {ji}}}.
\] The values \( s_{ij} \) can be quite general so long as \( s_{ij} = s_
{ji} \) for all \( i \) and \( j \) and \( 0 \le \alpha_{ij} \le 1 \).
With this choice of \( s_{ij} \) it is easy to verify that \( \pi \) is
the unique stationary distribution for \( P \).  For a symmetric \( Q \),
a simple choice of \( s_{ij} \) recovers the original Metropolis
algorithm.

For a given distribution \( \pi \), different choices of \( s_{ij} \)
give different Metropolis algorithms, all of which result in a Markov
chain with stationary distribution on \( \pi \).  P.\ H.\ Peskun showed
that among all choices of the \( s_{ij} \), the variance of the estimate
\[
    \frac{1}{M} \sum\limits_{\nu=1}^M f(s_\nu) \to \E{f}
\] is asymptotically minimal for the choice that leads to the Metropolis
algorithm.

\subsection*{Convergence} A basic problem of Markov chain theory but
especially for Markov chain Monte Carlo theory is:  What is the rate of
convergence in \( P^n(x,y) \to \pi(x) \)?  How long must the chain run
to be suitably close to \( \pi \)?  It is customary to measure distance
between probabilities by the total variation distance
\[
    \|P^n_x - \pi\|_{TV} = \frac{1}{2} \sum\limits_y \abs{ P^n_x - \pi }
    = \max_{A\subset X} \abs{ P^n(x,A) - \pi(A) }.
\] Probability distributions \( P^n_{x} \) and \( \pi \) are far apart
in total variation distance if there is a ``bad event'' \( A \) such
that \( P^n_{x} \) and \( \pi \) measure \( A \) differently.  The goal
is to show that for large \( n \) the Markov chain and the stationary
distribution do not share a ``bad event''.  This yields the problem:
Given \( P \), \( \pi \), \( x \), \( \epsilon > 0 \), how large must \(
n \) so that \( \| P^n_x - \pi \|_{TV} < \epsilon \)?  This question has
an answer for only a few problems, see for example the section on Card
Shuffling.

Suppose the Markov chain is \defn{reversible}, so \( \pi(x) P(x,y) = \pi
(y) P(y,x) \).%
\index{reversible Markov chain}
In words, reversibility means that at equilibrium the probability of
going from state \( x \) to state \( y \) is the same as going from \( y
\) to \( x \).  Here again the notation uses parentheses for matrix and
vector entries instead of subscripts.  The parenthesis notation
emphasizes the following connection to \( L^2 \) theory.  Let the space
be \( L^2(\pi) \), the set of mappings \( \{\vec{g} :  \mathcal{X} \to
\Reals \} \) with inner product \( \langle \vec{g},\vec{h} \rangle =
\sum_{x}g(x) h(x) \pi(x) \).  Take \( P \) to operate on \( L^2 \) by
usual matrix multiplication \( P\vec{g}(x) = \sum_y P(x,y) g(y) \).  The
property of reversibility implies \( \langle P\vec{g},\vec{h} \rangle =
\langle \vec{g},P \vec{h} \rangle \), so \( P \) is self-adjoint, see
the exercises.  The Spectral Theorem says the space has an orthonormal
basis (with respect to the inner product) of right (or column)
eigenvectors \( \vec{\psi}_i \) and real eigenvalues \( \beta_i \), so \(
P\vec{\psi}_i = \beta_i \vec{\psi}_i \) for \( 1 \le i \le \card{\mathcal
{X}} \) ordered by \( 1 = \beta_1 \ge \beta_2 \ge \cdots \ge \beta_{\card
{\mathcal{X}}} \ge -1 \).  Here subscripts are indexing the sets of
eigenvectors and eigenvalues, not entries of vectors or matrices.  Each
row of \( P \) sums to \( 1 \), so \( \vec{1} = (1,1, \dots, 1)^T \) is
a right eigenvector corresponding to eigenvalue \( 1 \).  Additionally,
with respect to the weighted inner product \( \| \vec{1} \|^{2} =
\langle \vec{1}, \vec{1} \rangle = \sum_{x} 1 \cdot 1 \cdot \pi(x) = 1 \),
so without loss of generality use \( \beta_1 = 1 \) and \( \psi_1 = \vec
{1} \) in the convergence estimate below.

By standard inner product space manipulations (see the exercises)
\begin{align*}
    P(x,y) &= \pi(y) \sum\limits_{\nu=1}^{\card{\mathcal{X}}} \beta_\nu
    \psi_\nu (x) \psi_\nu(y) \\
    P^n(x,y) &= \pi(y) \sum\limits_{\nu=1}^{\card{\mathcal{X}}} \beta_\nu^n
    \psi_\nu(x) \psi_\nu(y) \\
\end{align*}
Using the total variation metric definition and the Cauchy-Schwartz
inequality
\begin{align*}
    2\|P^n_x - \pi\|_{TV} &= \sum\limits_y \abs{P^n(x,y) - \pi(y) } \\
    &= \sum\limits_y \abs{P^n(x,y) - \pi(y) } \frac{\sqrt{\pi(y)}}{\sqrt
    {\pi(y)}}\\
    &\le \sqrt{\sum_y \frac{\abs{P^n(x,y) - \pi(y) }^2}{\left( \sqrt{\pi
    (y)} \right)^2}} \cdot \sqrt{\sum_y \left( \sqrt{\pi(y)} \right)^2}
    \\
    &= \sqrt{\sum_y \frac{\abs{P^n(x,y) - \pi(y)}^2}{\pi(y)} }.
\end{align*}

Then
\begin{align*}
    4 \|P^n_x - \pi \|^2_{\text{TV}} &\le \sum\limits_y \frac{\left(P^n(x,y)
    - \pi(y)\right)^2}{\pi(y)} \\
    &= \sum\limits_y \frac{\left(\pi(y) \sum\limits_{\nu=1}^{\card{\mathcal
    {X}}} \beta_\nu^n \psi_\nu(x) \psi_\nu(y) - \pi(y)\right)^2}{\pi(y)}
    \\
    &= \sum\limits_y \pi(y) \left(\sum\limits_{\nu=1}^{\card{\mathcal{X}}}
    \beta_\nu^n \psi_\nu(x) \psi_\nu(y) - 1 \right)^2.
\end{align*}
Expand the squared term, distribute the \( \pi(y) \) terms and
interchange the order of summation, factoring out terms not depending on
\( y \).
\begin{multline*}
    = \sum\limits_y \pi(y) \sum\limits_{\nu=1}^{\card{\mathcal{X}}}
    \beta_{\nu}^{2n} \psi_{\nu}(x)^2 \psi_{\nu}(y)^2 + 2 \sum_{\nu=1}^{\card
    {\mathcal{X}}} \sum_{\mu=\nu+1}^{\card{\mathcal{X}}} \beta_{\nu}
    \beta_{\mu} \psi_{\nu}(x)\psi_{\nu}(y) \psi_{\mu}(x) \psi_{\mu}(y)
    \\
    \qquad - 2 \sum_{\nu=1}^{\card{\mathcal{X}}} \beta_{\nu} \psi_{\nu}(x)\psi_
    {\nu}(y) + 1 \\
    = \sum\limits_y \sum\limits_{\nu=1}^{\card{\mathcal{X}}} \beta_{\nu}^
    {2n} \psi_{\nu}(x)^2 \psi_{\nu}(y)^2 \pi(y) + 2 \sum_{\nu=1}^{\card{\mathcal
    {X}}} \sum_{\mu=\nu+1}^{\card{\mathcal{X}}} \beta_{\nu} \beta_{\mu}
    \psi_{\nu}(x)\psi_{\mu}(y) \psi_{\mu}(x) \psi_{\mu}(y) \pi(y) \\
    \qquad - 2 \sum_{\nu=1}^{\card{\mathcal{X}}} \beta_{\nu} \psi_{\nu}(x)\psi_
    {\nu}(y)\pi (y) + \pi(y) \\
    = \sum\limits_{\nu=1}^{\card{\mathcal{X}}} \beta_{\nu}^{2n} \psi_{\nu}
    (x)^2 \sum\limits_y \psi_{\nu}(y)^2 \pi(y) \\
    + 2 \sum_{\nu=1}^{\card{\mathcal{X}}} \sum_{\mu=\nu+1}^{\card{\mathcal
    {X}}} \beta_{\nu} \beta_{\mu} \psi_{\nu}(x) \psi_{\mu}(x) \sum\limits_y
    \psi_{\nu}(y)\psi_{\mu}(y) \pi(y) \\
    \qquad - 2 \sum_{i=1}^{\card{\mathcal{X}}} \beta_i \psi_i(x) \sum\limits_y
    \psi_i(y)\pi(y) + \sum\limits_y \pi(y) \\
\end{multline*}
By orthonormality, \( \sum_y \psi_{\nu}(y)^2 \pi(y) = \langle \vec{\psi_
{\nu}}, \vec{\psi_\nu} \rangle = 1 \) for \( \nu = 1, \dots, \card{\mathcal
{X}} \). By orthogonality, \( \sum_y \psi_{\nu}(y) \psi_{\mu} \pi(y) =
\langle \vec{\psi_{\nu}}, \vec{\psi_{\mu}} \rangle = 0 \) for \( \nu =
1, \dots, \card{\mathcal{X}} \) and \( \mu = \nu+1, \dots, \card{\mathcal
{X}} \).  Recalling \( \beta_1 = 1 \) and \( \psi_1 = \vec{1} \), by
orthogonality \( \sum_y \psi_i(y) \pi(y) = \sum\limits_y 1 \cdot \psi_i(y)
\pi(y) = \langle \vec{\psi_1}, \vec{\psi_i} \rangle = 0 \) for \( i = 2,
\dots \card{\mathcal{X}} \) while \( \sum_y \psi_1(y) \pi(y) = 1 \).
Also, \( \sum_y \pi(y) = 1 \).  This leaves
\begin{align*}
    & = \beta_1^{2n} \psi_1(x)^2 + \sum\limits_{i=2}^{\card{\mathcal{X}}}
    \beta_i^{2n} \psi_i(x)^2 - 2 \beta_1^n \psi_1(x) + \sum\limits_y \pi
    (y) \\
    &= 1 + \sum\limits_{i=2}^{\card{\mathcal{X}}} \beta_i^{2n} \psi_i(x)^2
    - 2 + 1 \\
    &= \sum\limits_{i=2}^{\card{\mathcal{X}}} \beta_i^{2n} \psi_i(x)^2.
\end{align*}
Summarizing
\begin{equation}
    \|P^n_x - \pi \|^2_{\text{TV}} \le \frac{1}{4} \sum\limits_{i=2}^{\card
    {\mathcal{X}}} \beta_i^{2n} \psi_i(x)^2.%
    \label{eq:montecarlo:bound}
\end{equation}

This bound is the basic eigenvalue bound used to get rates of
convergence for the Markov chain Monte Carlo method.  Recalling Theorem
11 from Stationary Distributions, there exist constants \( \alpha \in (0,1)
\) and \( C > 0 \) such that
\[
    \max_{x \in \mathcal{X}} \| (P^n)_{i \cdot} - \pi \|_{TV} \le C
    \alpha^n.
\] The results here give more information, but to get good bounds on the
right hand side requires knowledge of both eigenvalues and eigenvectors.

\subsection*{A Simple Example}

Recall the magical land of Oz where the weather follows a pattern.  If
it is raining today (R), then tomorrow has a \( 50\% \) chance of
raining again, and a \( 25\% \) chance of either having a nice day (N)
or a snowy day (S) tomorrow.  Similarly if it is snowing, we have a \(
50\% \) chance of again having snow, and a \( 25\% \) chance of either
having a nice day, or a rainy day.  Also the land of Oz never has two
nice days in a row, and equally has rain or snow the day after a nice
day.  Further recall that for \( P^n \), for large values of \( n \),
the row vectors approach the stationary distribution \( \mathbf{\pi}=(2/5,1/5,2/5)
\) in the order \( (R, N, S) \).

Take this distribution as the desired distribution \( \pi \).  To
illustrate the Markov chain Monte Carlo method, use the
Metropolis-Hastings algorithm.  For simplicity and in the absence of any
other reasonable transition probability matrix, take the initial
transition probability matrix to be uniform on the state space of
weather days, \( P = (p_{ij}) \equiv \frac{1}{3} \).  Then \( \alpha_{ij}
\) is given by
\[
    \alpha_{ij} = \frac{s_{ij}}{1 + \frac{\pi_i}{\pi_j} \frac{p_{ij}}{p_
    {ji}}}
\] with \( s_{ij} \equiv 1 \) again for simplicity and the lack of any
other reasonable choice.  With these choices,
\[
    \alpha =
    \begin{pmatrix}
        \frac{1}{2} & \frac{1}{3} & \frac{1}{2} \\
        \frac{2}{3} & \frac{1}{2} & \frac{2}{3} \\
        \frac{1}{2} & \frac{1}{3} & \frac{1}{2} \\
    \end{pmatrix}
    .
\] Following the construction of the Metropolis-Hastings algorithm
\[
    Q =
    \begin{pmatrix}
        \frac{13}{18} & \frac{1}{9} & \frac{1}{6} \\
        \frac{2}{9} & \frac{5}{9} & \frac{2}{9} \\
        \frac{1}{6} & \frac{1}{9} & \frac{13}{18}
    \end{pmatrix}
    .
\] Then \( \pi = (2/5, 1/5, 2/5) \) is a stationary distribution of \( Q
\), just as it is for \( P \).  Comparing the rates of convergence is
interesting too.  The eigenvalues of \( P \) are \( 1, 1/4, -1/4 \), so
the rate of convergence of \( P^n \) is about like \( (1/4)^n \).  On
the other hand, the eigenvalues of \( Q \) are \( 1, 5/9, 4/9 \), so the
rate of convergence of \( Q^{n} \) is about like \( (5/9)^n \), much
slower.  This is not surprising, since sampling the stationary
distribution from the Markov chain is not as direct.

The Markov chain with \( Q \) as transition probability matrix is
reversible.  After \( 5 \) steps the square of the total variation
distance of the distribution from state \( 1 \), Rain, from the
stationary distribution is approximately \( 7.95 \times 10^{-4} \) while
the bound from~\eqref{eq:montecarlo:bound} is \( 8.94 \times 10^ {-4} \).
After \( 10 \) steps the square of the total variation distance of the
distribution from state \( 1 \) from the stationary distribution is
approximately \( 2.05 \times 10^{-6} \) while the bound from~\eqref{eq:montecarlo:bound}
is \( 2.45 \times 10^{-6} \).  Regression analysis gives
\[
    \| (P^n)_{1, \cdot} - \pi \|_{TV} \approx 0.5295 \cdot (0.5538)^n,
\] about what is expected.

\subsection*{Example with Continuous State Space}

Consider three particles each randomly located in the square \( [0,1]^2
\subset \Reals^2 \) with positions \( (x_{i1}, x_{i2}) \) for \( i =
1,2,3 \). The state space is \( \mathcal{X} = [0,1]^6 \).  Suppose the
positions of the particles are distributed according to the density
\[
    \pi(x) = \pi(x_1,x_2,x_3) = \frac{1}{z} \exp\left[ -C \sum\limits_{i=1}^3
    \|x_i\| - D \sum\limits_{i<j} \frac{1}{\| x_i - x_j \|} \right]
\] where \( \| \cdot \| \) is the usual Euclidean norm on \( \Reals^2 \),
\( C \) and \( D \) are fixed positive constants and \( z \) is the
normalizing constant or partition function.  In this density the first
sum pushes the particle towards the origin, the second sum pushes them
away from each other.

To create a Markov chain with \( \pi \) as its stationary distribution,
use the Metropolis algorithm. Given \( X_n \), first choose \( y \in [0,1]^6
\) from the uniform distribution on \( \mathcal{X} \).  Then with
probability \( r = \min[1, \pi(y)/\pi(x)] \), accept \( X_{n+1} = y \),
otherwise reject \( y \) and \( X_{n+1} = x \).  Then this Markov chain
has \( \pi \) as its stationary distribution.  The later section on
Convergence to the Stationary Distribution will show that with \( C = D
= \frac{1}{10} \) the rate of convergence is
\[
    \|
    \operatorname{dist}
    (X_n) - \pi \|_{TV} \le (0.883)^n
\] so that after \( 38 \) steps, the total variation distance between
the Markov chain and the stationary distribution is less than \( 0.01 \).

A plot of the stationary distribution is Figure~%
\ref{fig:montecarlo:pointproc}.

\begin{figure}
    \centering

    \caption{Plot of points \( x_1 \), \( x_2 \), \( x_3 \) from step \(
    100 \) to step \( 3000 \) in the Metropolis algorithm simulation of
    the stationary distribution \( \pi \) for the continuous state space
    Markov chain.  }%
    \label{fig:montecarlo:pointproc} \includegraphics{pointproc}

\end{figure}

\subsection*{Example with Unbounded Continuous State Space}

Let the state space be \( \mathcal{X} = \Reals \) with target density
the double exponential or Laplace distribution. \( \pi(x) = \EulerE^{-\abs
{x}}/2 \).  It is easy to simulate the double exponential density using
the standard technique of choosing a uniform random deviate and then
applying the inverse of the cumulative distribution function, so this is
an illustrative example rather than a practical simulation method for
the double exponential distribution.

Use another version of the Metropolis algorithm.  Choose the initial
state \( x_0 \) uniformly at random from the interval \( [ -2, +2] \).
At succeeding steps, first propose to move from state \( x \) to \( y \)
chosen uniformly at random from the interval \( [x - 2, x + 2] \).  With
probability \( \min[1, \pi(y)/\pi(x)] \), accept the proposal \( y \)
which becomes the new state, otherwise reject it and remain at \( x \).
Again, this procedure creates a Markov chain with \( \pi \) as its
stationary distribution.

The later section on Convergence to the Stationary Distribution will
show that a theoretical upper bound on the rate of convergence is
\[
    \|
    \operatorname{dist}
    (X_n) - \pi \|_{TV} \le (0.983)^{1 + n/439.56} + (2 \cdot (20.04)^{n/439.56})
    (0.993)^n
\] so that after \( 120{,}000 \) steps, the total variation distance
between the Markov chain and the stationary distribution is less than \(
0.01 \).  The R script in the Algorithms section shows that after about \(
48{,}000 \) steps total variation distance between this Markov chain and
the stationary distribution is less than \( 0.05 \).  Either way, the
convergence to a stationary distribution is much slower than other
examples.

% \subsection*{An Example from Symmetric Functions}

% Let $\mathcal{X} = S_n$, the symmetric group on $n$ letters.  Define a
% probability measure on $S_n$ by
% \[
%     \pi(\sigma) = z^{-1}\theta^{d(\sigma, \sigma_0)}
% \]
% for $\sigma, \sigma_0 \in S_n$ and $0 < \theta \le 1$.  The function
% $d(\sigma, \sigma_0)$ is a metric on the symmetric group,
% \[
%     d(\sigma,\sigma_0) = \text{ minimum number of transpositions
%       required to bring $\sigma$ to $\sigma_0$. }
% \]
% This metric is called \emph{Cayley's distance} because a result of
% Cayley implies that $d(\sigma,\sigma_0) = n - c(\sigma^{-1}\sigma_0)$
% with $c(\tau)$ being the number of cycles in $\tau$.
% %% https://rdrr.io/rforge/Rankcluster/man/distCayley.html
% %% https://www.findstat.org/StatisticsDatabase/St000216

% The metric is
% bi-invariant, $d(\sigma,\sigma_0) = d(\tau\sigma, \tau\sigma_0) =
% d(\sigma\tau, \sigma_0\tau)$.  The normalizing constant $z$ is known
% to be $z = \sum\limits_{\sigma \in S_n} \theta^{d(\sigma,\sigma_0)} =
% \prod\limits_{i=1}^n (1 + \theta \cdot (i-1))$.
% If $\theta=1$, $\pi(\sigma)$ is the uniform distribution on $S_n$.
% For $\theta < 1$, $\pi(\sigma)$ is largest at $\sigma_0$ and falls off
% from its maximum as $\sigma$ moves away from $\sigma_0$.  The problem
% here is \emph{How can samples be drawn from $S_n$ with this
%   probability distribution?}

% The example answers this question with the Metropolis algorithm based
% on random transposition.  From $\sigma \in S_n$, choose a
% transposition $(i,j)$ uniformly at random and consider $(i,j)\sigma =
% \sigma^{\star}$.

\visual{Section Starter Question}{../../../../CommonInformation/Lessons/question_mark.png}
\section*{Section Ending Answer}

In statistical mechanics and mathematics, a Boltzmann distribution is a
probability distribution or probability measure that gives the
probability that a system will be in a certain state or
``configuration'' as a function of that state's energy and the
temperature of the system.  The distribution is expressed in the form:
\[
    p_{i}\propto \EulerE^{-{\frac {\varepsilon _{i}}{kT}}}
\] where \( p_i \) is the probability of the system being in state \( i \),
\( \varepsilon_i \) is the energy of that state, and a constant \( kT \)
of the distribution is the product of Boltzmann's constant \( k \) and
thermodynamic temperature \( T \).  The distribution shows that states
with lower energy will always have a higher probability of being
occupied.

\subsection*{Sources} This section is adapted from:  ``The Evolution of
Markov Chain Monte Carlo Methods'' by Matthew Richey
\cite{richey10} and ``The Markov chain Monte Carlo revolution'' by Persi
Diaconis
\cite{diaconis09}.  The examples of the continuous state space and the
unbounded state space is adapted from
\cite{jiang21}.

\hr

\visual{Algorithms, Scripts, Simulations}{../../../../CommonInformation/Lessons/computer.png}
\section*{Algorithms, Scripts, Simulations}

\subsection*{Algorithm}

\begin{algorithm}[H]
  \DontPrintSemicolon
  \KwData{Desired stationary distribution}
  \KwResult{Approximate stationary distribution}
  \BlankLine
  \emph{Initialization of Matrices}\;
     Load Markov chain library\;
     Set number of states\;
     Set simple uniform transition probability matrix\;
     Set state names\;
     Set desired stationary distribution\;
     Set up matrix with \( s \) values\;
     Set matrix with \( \alpha \) values according to algorithm\;
     Set new probability transition matrix\;
     Set markovchain object\;
     Set length of long sample path and start position for sample\;
     Slice the long sample path from the transient time to the end\;
     In the slice count the appearance of each state\;
     Empirical stationary distribution\;
  \caption{Metropolis-Hastings algorithm}
  \end{algorithm}
  
\subsection*{Scripts}


\begin{description}


\item[R] 

\link{http://www.math.unl.edu/~sdunbar1/metHast.R}{R script for metHast.R}

\begin{lstlisting}[language=R]
library("markovchain")

nStates <- 3
P <- matrix(rep(1/nStates, nStates * nStates),
             nrow=nStates, byrow=TRUE)
stateNames <- c("Rain", "Nice", "Snow")
rownames(P) <- stateNames
colnames(P) <- stateNames

pi <-  c(2/5, 1/5, 2/5)
s <- matrix(rep(1, nStates * nStates),
            nrow=nStates, byrow=TRUE)

alpha <- matrix(0, nStates,nStates)
for (i in 1:nStates) {
    for (j in 1:nStates) {
        alpha[i, j] <- s[i, j] * (1 /( 1 + (pi[i]/pi[j])*(P[i,j]/P[j,i])))
    }
}

Q <- matrix(0, nStates,nStates)
for (i in 1:nStates) {
    for (j in 1:nStates) {
        if (i != j) {
            Q[i, j] <- P[i, j] * alpha[i, j]
        }
    }
    Q[i, i] <- 1 - sum( Q[i, ] ) 
}

hastMetWeather <- new("markovchain", states = stateNames,
                   transitionMatrix = Q,
                 name = "hastMetWeather")
cat( "Steady State Distribution: ", steadyStates(hastMetWeather), "\n" )

start <- 900
end <- 1000
hastMetWeatherSim <- rmarkovchain(n=end, object=hastMetWeather,
                                  t0="Nice")
cat("Simulation Results\n")
cat( "Rain:", sum( hastMetWeatherSim[(start+1):end] == "Rain")/(end-start), "\n")
cat( "Nice:", sum( hastMetWeatherSim[(start+1):end] == "Nice")/(end-start), "\n")
cat( "Snow:", sum( hastMetWeatherSim[(start+1):end] == "Snow")/(end-start), "\n")
\end{lstlisting}  

\end{description}



%%% Local Variables:
%%% TeX-master: t
%%% End:


\hr

\visual{Problems to Work}{../../../../CommonInformation/Lessons/solveproblems.png}
\section*{Problems to Work for Understanding}
\renewcommand{\theexerciseseries}{}
\renewcommand{\theexercise}{\arabic{exercise}}

\begin{exercise}
    Consider the following Markov Chain:  \( N \) black balls and \( N \)
    white balls are placed in two urns so that each contains \( N \)
    balls.  At each step one ball is selected at random from each urn
    and the two balls interchange urns.  Let \( N = 3 \) and the state
    of the system is the number of white balls in the first urn.
    \begin{enumerate}[label=(\alph*)]
    \item
        What is the probability transition matrix \( P \)?
    \item
        What is the stationary distribution?
    \item
        What is the matrix \( \alpha \) of entries \( \alpha_{ij} \) in
        the Metropolis-Hastings algorithm?
    \item
        Choose a satisfactory transition matrix and then create the
        matrix \( Q \) in the Metropolis-Hastings algorithm.
    \item
        Find the eigenvalues of \( P \) and the approximate rate of
        convergence to the stationary distribution.
    \item
        Find the eigenvalues of \( Q \) and the approximate rate of
        convergence to the stationary distribution.
\end{enumerate}
\end{exercise}
\begin{solution}
    \begin{enumerate}[label=(\alph*)]
    \item
        For \( N = 3 \), the transition probability matrix is
        \[
            \begin{pmatrix}
                0 & 1 & 0 & 0 \\
                1/9 & 4/9 & 4/9 & 0 \\
                0 & 4/9 & 4/9 & 1/9 \\
                0 & 0 & 1 & 0
            \end{pmatrix}
            .
        \]
    \item
        The stationary distribution is \( \pi = (1/20, 9/20, 9/20, 1/20)
        \).
    \item
        For simplicity and without prior knowledge on the transitions
        among the \( 4 \) states choose
        \[
            \begin{pmatrix}
                1/4 & 1/4 & 1/4 & 1/4 \\
                1/4 & 1/4 & 1/4 & 1/4 \\
                1/4 & 1/4 & 1/4 & 1/4 \\
                1/4 & 1/4 & 1/4 & 1/4 \\
            \end{pmatrix}
            .
        \] Then the denominator of the \( \alpha_{ij} \) values is
        \[
            \alpha_{ij} = \frac{1}{1 + \frac{\pi_i}{\pi_j} \frac{p_ {ij}}
            {p_{ji}}} = \frac{1}{1 + \frac{\pi_i}{\pi_j}}.
        \] The matrix of \( \alpha_{ij} \) values is
        \[
            \begin{pmatrix}
                \frac{1}{2} & \frac{9}{10} & \frac{9}{10} & \frac{1}{2}\\
                \frac{1}{10} & \frac{1}{2} & \frac{1}{2} & \frac{1}{10}\\
                \frac{1}{10} & \frac{1}{2} & \frac{1}{2} & \frac{1}{10}\\
                \frac{1}{2} & \frac{9}{10} & \frac{9}{10} & \frac{1}{2}
            \end{pmatrix}
            .
        \] Since all entries are between \( 0 \) and \( 1 \), choose \(
        s_{ij} =1 \), and the matrix above is the desired \( Q \).  Then
        the matrix for the Metropolis-Hastings algorithm is
        \[
            \begin{pmatrix}
                \frac{17}{40} & \frac{9}{40} & \frac{9}{40} & \frac{1}{8}\\
                \frac{1}{40} & \frac{33}{40} & \frac{1}{8} & \frac{1}{40}\\
                \frac{1}{40} & \frac{1}{8} & \frac{33}{40} & \frac{1}{40}\\
                \frac{1}{8} & \frac{9}{40} & \frac{9}{40} & \frac{17}{40}
            \end{pmatrix}
            .
        \] This has the same stationary distribution.
    \item
        The eigenvalues of \( P \) are \( 1, 1/3, -1/3, -1/9 \).  The
        rate of convergence to the stationary distribution is
        approximately \( (1/3)^n \).
    \item
        The eigenvalues of \( Q \) are \( 1, 7/10, 1/2, 3/10 \).  The
        rate of convergence to the stationary distribution is
        approximately \( (7/10)^n \), much slower.
\end{enumerate}
\end{solution}
\begin{exercise}
    Consider the second Ehrenfest urn model with \( N=4 \), \( p = q =
    \frac{1}{2} \) and transition probability matrix
    \[
        P =
        \begin{pmatrix}
            \frac{1}{2} & \frac{1}{2} & 0 & 0 & 0\\
            \frac{1}{8} & \frac{1}{2} & \frac{3}{8} & 0 & 0\\
            0 & \frac{1}{4} & \frac{1}{2} & \frac{1}{4} & 0\\
            0 & 0 & \frac{3}{8} & \frac{1}{2} & \frac{1}{8}\\
            0 & 0 & 0 & \frac{1}{2} & \frac{1}{2}
        \end{pmatrix}
        .
    \]
    \begin{enumerate}[label=(\alph*)]
    \item
        Show that this Markov chain is irreducible, positive recurrent,
        aperiodic.
    \item
        Find the stationary distribution \( \pi \).
    \item
        Show the Markov chain is reversible.
    \item
        Find the eigenvalues and corresponding eigenvectors, ordered by
        decreasing eigenvalue.
    \item
        Using the inner product weighted by \( \pi \), find the
        normalized orthonormal eigenvector basis.
    \item
        Show that
        \[
            P(x,y) = \pi(y) \sum\limits_{i=1}^{5} \beta_i \psi_i(x) \psi_i
            (y).
        \]
\end{enumerate}
\end{exercise}
\begin{solution}
    \begin{enumerate}[label=(\alph*)]
    \item
        Since each state has a positive probability of transitioning to
        its neighbor state, the chain is irreducible.  Alternatively,
        each entry in \( P^4 \) is non-zero, so the chain is
        irreducible.  Since this is a finite state space irreducible
        chain, it is positive recurrent.  Since each state has a
        positive probability of staying in the state, the minimum return
        time is \( 1 \), so the chain is aperiodic.
    \item
        \[
            \pi = \left( \frac{1}{16}, \frac{1}{4}, \frac{3}{8}, \frac{1}
            {4}, \frac{1}{16} \right).
        \]
    \item
        It is easy to check the reversibility, but both the physical
        description, and the symmetry of \( \pi \) and \( P \) suggest
        that reversibility holds.
    \item
        The eigenvalues, ordered by decreasing value, are \(
        1,3/4,1/2,1/4,0 \).
    \item
        The corresponding eigenvectors, without normalization, are
        \begin{multline*}
            (1,1,1,1,1)^T,(1,1/2,0,-1/2,-1)^T,(1,0,-1/3,0,1)^T, \\
            (1,-1/2,0,1/2,-1)^T, (1,-1,1,-1,1)^T.
        \end{multline*}
    \item
        The weighted inner products of each eigenvector with itself is
        respectively \( 1, 1/4, 1/6, 1/4, 1 \) so the normalized
        eigenvectors are
        \begin{multline*}
            \left( 1,1,1,1,1\right)^T, \left( 2,1,0,-1,-2\right)^T,
            \left( \sqrt{6},0,-\frac{\sqrt{6}}{3},0,\sqrt{6}\right)^T,
            \\
            \left( 2,-1,0,1,-2\right)^T, \left( 1,-1,1,-1,1\right)^T.
        \end{multline*}
    \item
        With (for example) the Maxima command
\begin{verbatim}
      generateP: genmatrix(lambda([i,j], pi[j] *
     sum(beta[k]*psi[k][i]*psi[k][j], k, 1, 5)[1]), 5, 5);
\end{verbatim}
        the result is
        \[
            P =
            \begin{pmatrix}
                \frac{1}{2} & \frac{1}{2} & 0 & 0 & 0\\
                \frac{1}{8} & \frac{1}{2} & \frac{3}{8} & 0 & 0\\
                0 & \frac{1}{4} & \frac{1}{2} & \frac{1}{4} & 0\\
                0 & 0 & \frac{3}{8} & \frac{1}{2} & \frac{1}{8}\\
                0 & 0 & 0 & \frac{1}{2} & \frac{1}{2}
            \end{pmatrix}
            .
        \] as expected.
\end{enumerate}
\end{solution}
\begin{exercise}
    \begin{enumerate}[label=(\alph*)]
    \item
        Using the results from the previous problem, find the upper
        bound~\eqref{eq:montecarlo:bound} for the square of the total
        variation distance of the distribution starting from state \( 5 \),
        with all balls in the first urn, from the stationary
        distribution.
    \item
        With numerical computation, check the inequality for \( 5 \) and
        \( 10 \) steps of the Markov chain.
\end{enumerate}
\end{exercise}
\begin{solution}
    \begin{enumerate}[label=(\alph*)]
    \item
        Using the normalized eigenvectors from the previous solution,
        the upper bound is:
        \begin{multline*}
            \frac{1}{4} \sum\limits_{i=2}^{\card{\mathcal{X}}} \beta_i^{2n}
            \psi_i(x)^2 \\
            = \frac{1}{4} \left[ \left(\frac{3}{4} \right)^{2n} \cdot (-2)^
            {2} + \left( \frac{1}{2} \right)^{2n} \cdot (\sqrt {6})^2 +
            \left(\frac{1}{4} \right)^{2n} \cdot (-2)^{2} + (0)^ {2n} (-2)^
            {2} \right] \\
            = \left( \frac{3}{4} \right)^{2n} + \frac{3}{2} \cdot \left(
            \frac{1}{2} \right)^{2n} + \left( \frac{1}{4} \right)^{2n}
        \end{multline*}
    \item
        After 5 steps, the total variation distance squared is
        approximately \( 0.0359 \) and the upper bound is approximately \(
        0.0577 \) so the inequality bound holds.  After 10 steps, the
        total variation distance squared is approximately \( 0.00181 \)
        and the upper bound is approximately \( 0.00317 \) so the
        inequality bound holds.
\end{enumerate}

\end{solution}
\begin{exercise}
    Suppose the Markov chain is reversible, so \( \pi(x) P(x,y) = \pi(y)
    P(y,x) \).  Let the space be \( L^2(\pi) \), the set of mappings \(
    \{g :  X \to \Reals \} \) with inner product \( (g,h) = \sum\limits_
    {x}g(x) h(x) \pi(x) \).  Then \( P \) operates on \( L^2 \) by
    multiplication \( Pg(x) = \sum\limits_y P(x,y) g(y) \).  Show that
    the property of reversibility implies \( (Pg,h) = (g,Ph) \), so \( P
    \) is self-adjoint.
\end{exercise}
\begin{solution}
    \begin{align*}
        \langle Pg, h \rangle &= \sum_x \sum_y P(x,y)g(y) h(x) \pi(x) \\
        &= \sum_x \sum_y g(y) h(x) \pi(x) P(x,y) \\
        &= \sum_x \sum_y g(y) h(x) \pi(y) P(y,x) \\
        &= \sum_x \sum_y g(y) P(y,x)h(x) \pi(y) \\
        &= \sum_y \sum_x g(y) P(y,x)h(x) \pi(y) \\
        &= \sum_y g(y) \sum_x P(y,x)h(x) \pi(y) \\
        &= \langle g, Ph \rangle.
    \end{align*}
\end{solution}
\begin{exercise}
    For self-adjoint matrix \( P \) and stationary distribution \( \pi \),
    with an orthonormal basis (with respect to the inner product \( (g,h)
    = \sum\limits_{x}g(x) h(x) \pi(x) \)) with right eigenvectors \(
    \psi_i \) and real eigenvalues \( \beta_i \), so \( P\psi_i = \beta_i
    \psi_i \) for \( 1 \le i \le \card{\mathcal{X}} \) ordered by \( 1 =
    \beta_1 \ge \beta_2 \ge \cdots \ge \beta_{\card{\mathcal{X}}} \ge -1
    \), show that
    \[
        P(x,y) = \pi(y) \sum\limits_{i=1}^{\card{\mathcal{X}}} \beta_i
        \psi_i(x) \psi_i(y).
    \]
\end{exercise}
\begin{solution}
    Let \( \vec{e}_x \) be the elementary column vector with \( 1 \) at
    position \( x \) and \( 0 \) elsewhere.  Since \( \vec{e}_x \) is a
    probability distribution concentrated entirely on state \( x \),
    then \( \vec{e}^T_x \cdot P \) is the probability distribution of
    states after one step from \( x \).  Then \( \vec{e}^T_x \cdot P
    \cdot \vec{e}_y \) picks out \( P(x,y) \), that is, \( \vec{e}^T_x
    \cdot P \cdot \vec{e}_y = P(x,y) \).  The point is that this
    probability product does not involve the inner product weight vector
    \( \pi \).

    The orthonormal eigenvector basis represents \( \vec{e}_x \) in
    terms of \( \vec{\psi}_i \) as \( \vec{e}_x = \sum_i a_{xi} \vec{\psi}_i
    \) for some scalars \( a_{xi} \).  Projecting \( \vec{e}_x \)
    against \( \vec{\psi}_i \) with the weighted inner product
    \[
        a_{xi} = \langle \vec{e}_{x}, \vec{\psi}_{i} \rangle = \psi_i(x)
        \pi(x)
    \] so as a column vector
    \[
        \vec{e}_x =\sum\limits_i \psi_i(x) \pi(x) \vec{\psi}_i.
    \]

    Using
    \[
        P(x,y) = \vec{e}^T_x \cdot P \cdot \vec{e}_y
    \] where \( \vec{e}^T_x \cdot P \cdot \vec{e}_y \) indicates the
    usual matrix-vector product with rows dotted against columns, and
    expressing \( \vec{e}_y \), but not \( \vec{e}_x \), in terms of the
    orthonormal basis
    \begin{align*}
        & = \vec{e}_x \cdot P \cdot \left( \sum\limits_i \psi_i(y) \pi(y)
        \vec{\psi}_i \right) \\
        & = \vec{e}_x \cdot \left( \sum\limits_i \psi_i(y) \pi(y) P
        \cdot \vec{\psi}_i \right) \\
        &= \vec{e}_x \cdot \left( \sum\limits_i \psi_i(y) \pi(y) \beta_i
        \vec{\psi}_i \right) \\
    \end{align*}
    using the eigenvector property.  Now factoring the dot product with \(
    \vec{e}_x \) through the sum will pick out the \( x \)th term from
    each \( \vec{\psi}_i \),
    \begin{align*}
        &= \sum\limits_i \psi_i(y) \pi(y) \beta_i \left( \vec{e}_x \cdot
        \vec{\psi}_i \right) \\
        & = \sum\limits_i \psi_i(y) \pi(y) \beta_i \psi_i(x)
    \end{align*}
    and finally, \( \pi(y) \) is independent of the summation over \( i \)
    so it can be factored out of the sum, and rearranging the products
    yields
    \[
        = \pi(y) \sum\limits_i \beta_i \psi_i(x) \psi_i(y).
    \]
\end{solution}

\visual{Books}{../../../../CommonInformation/Lessons/books.png}
\section*{Reading Suggestion:}

\bibliography{../../../../CommonInformation/bibliography}

%   \begin{enumerate}
%     \item
%     \item
%     \item
%   \end{enumerate}

\hr

\visual{Links}{../../../../CommonInformation/Lessons/chainlink.png}
\section*{Outside Readings and Links:}
\begin{enumerate}
    \item
    \item
    \item
    \item
\end{enumerate}

\section*{\solutionsname} \loadSolutions

\hr

\mydisclaim \myfooter

Last modified:  \flastmod

\end{document}


%%% Local Variables:
%%% TeX-master: t
%%% TeX-master: t
%%% End:
